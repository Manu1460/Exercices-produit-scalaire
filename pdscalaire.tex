\documentclass[12pt,a4paper]{article}
\usepackage[utf8]{inputenc}
\usepackage[T1]{fontenc}
\usepackage{amsmath}
\usepackage{amsfonts}
\usepackage{amssymb}
\usepackage{multicol}
\usepackage{qrcode}
\usepackage{lmodern}
\usepackage{colortbl}%permet de griser les cases
\usepackage{tabularx, multirow}
%\usepackage{lscape}
\usepackage{xcolor}
%\usepackage{graphicx}
\usepackage{tikz,tkz-base}
% Fichier de style stage.sty [UTF8]
% Copyleft Laurent Bretonnière, laurent.bretonniere@gmail.com
% Version du 14/03/2015 

%*********************************************************************************************
% Packages
%*********************************************************************************************

\usepackage[utf8]{inputenc}%			encodage du fichier source (Linux)
\usepackage[TS1,T1]{fontenc}%			gestion des accents (pour les pdf)
\usepackage[french]{babel}%				rajouter éventuellement greek, etc.
\frenchbsetup{CompactItemize=false,StandardLists=true}
\usepackage{enumitem}%
\setenumerate[1]{label=\arabic*/}%
\setenumerate[2]{label=\alph*/}%
%\setlist{font=\bfseries,leftmargin=*}%
\setlist{font=\bfseries,leftmargin=*,topsep=1pt,partopsep=1pt,itemsep=2pt,parsep=1pt}%

\usepackage{textcomp}%					caractères additionnels
\usepackage{amsmath,amssymb}%			pour les maths (1)
\usepackage{amsfonts}%					pour les maths (2)
\usepackage{lmodern}%					remplacer éventuellement par txfonts, fourier, etc.

\usepackage{graphicx}%

% cf site web http://www.khirevich.com/latex/microtype/
%\usepackage[babel=true,kerning=true]{microtype}%
\usepackage{microtype}%

\usepackage{dsfont}%					pour les ensembles de nombres N,Z,D,Q,R,C ...
\usepackage{mathrsfs}%					pour les écritures calligraphiques (genre Cf et Df)
\usepackage[np]{numprint}% page 58  	pour afficher les nombres 3 par 3
\usepackage[e]{esvect}% page 60		vecteurs
\usepackage{stmaryrd}%					pour les intervalles entiers et sslash
\usepackage{empheq}% 					encadrer en mode math
\usepackage{xcolor}%					pour gérer les couleurs
\usepackage{soul}% 						pour les fluos	
\usepackage{xspace}%					gestion des espaces
\usepackage{pifont}% 					trèfle, pique, carreau, coeur
\usepackage{eurosym}%					symbole euro
\usepackage{mathabx}%					choix personnel

\usepackage{hyperref}%
\hypersetup{%
colorlinks=true,%
breaklinks=true,%
citecolor=red,%
urlcolor=blue,%
linkcolor=black,% ou blue
bookmarksopen=false,%
pdfcreator=PDFLaTeX,%
pdfproducer=PDFLaTeX,%
pdfmenubar=true,%
pdftoolbar=true,%
pdfauthor={Laurent Bretonnière},%
pdfkeywords={Mathématiques},%
pdfstartview=XYZ}%

%*********************************************************************************************
% Mathématiques (cf chapitre 7 page 56, LaTeX pour le prof de maths, IREM de Lyon)
%*********************************************************************************************

% Fonctions usuelles
\DeclareMathOperator{\cotan}{cotan}%
\DeclareMathOperator{\ch}{ch}%
\DeclareMathOperator{\sh}{sh}%
\DeclareMathOperator{\thyp}{th}%
\renewcommand{\th}{\thyp}%
\DeclareMathOperator{\Arcsin}{Arcsin}%
\DeclareMathOperator{\Arccos}{Arccos}%
\DeclareMathOperator{\Arctan}{Arctan}%
\DeclareMathOperator{\Argsh}{Argsh}%
\DeclareMathOperator{\Argch}{Argch}%
\DeclareMathOperator{\Argth}{Argth}%
\DeclareMathOperator{\pgcd}{pgcd}%
\DeclareMathOperator{\ppcm}{ppcm}%
\DeclareMathOperator{\card}{card}%

% Composée de fonctions
\newcommand{\rond}{\circ}%

% Multiplication
\newcommand{\x}{\times}

% Constantes usuelles
\renewcommand{\i}{\mathrm{i}}%
\newcommand{\e}{\mathrm{e}}%

% Éléments différentiels
\newcommand{\dt}{\,\textrm{d}t}%
\newcommand{\du}{\,\textrm{d}u}%
\newcommand{\dv}{\,\textrm{d}v}%
\newcommand{\dw}{\,\textrm{d}w}%
\newcommand{\dx}{\,\textrm{d}x}%
\newcommand{\dy}{\,\textrm{d}y}%
\newcommand{\dz}{\,\textrm{d}z}%

% Ensembles de nombres
\newcommand{\ensnb}[1]{\ensuremath{\mathbb{#1}}}%
\newcommand{\N}{\ensnb{N}}%
\newcommand{\Z}{\ensnb{Z}}%
\newcommand{\D}{\ensnb{D}}%
\newcommand{\Q}{\ensnb{Q}}%
\newcommand{\R}{\ensnb{R}}%
\newcommand{\Rp}{\R_{+}}%
\newcommand{\Rm}{\R_{-}}%
\newcommand{\mtsmall}{\fontsize{5pt}{5pt}\selectfont}%
\newcommand{\Rpe}{\R_{\mbox{\mtsmall$+$}}^{\mskip0.4mu\ast}}%
\newcommand{\Rme}{\R_{\mbox{\mtsmall$-$}}^{\mskip0.4mu\ast}}%
\newcommand{\Ret}{\R^{\ast}}% \Re pris pour la partie réelle d'un complexe
\newcommand{\Ne}{\N^{\ast}}%
\newcommand{\Ze}{\Z^{\ast}}%
\newcommand{\C}{\ensnb{C}}%
\newcommand{\Ce}{\C^{\ast}}%

% Noms de points en majuscules en romain (et non pas en italiques)
\DeclareMathSymbol{A}{\mathalpha}{operators}{`A}
\DeclareMathSymbol{B}{\mathalpha}{operators}{`B}
\DeclareMathSymbol{C}{\mathalpha}{operators}{`C}
\DeclareMathSymbol{D}{\mathalpha}{operators}{`D}
\DeclareMathSymbol{E}{\mathalpha}{operators}{`E}
\DeclareMathSymbol{F}{\mathalpha}{operators}{`F}
\DeclareMathSymbol{G}{\mathalpha}{operators}{`G}
\DeclareMathSymbol{H}{\mathalpha}{operators}{`H}
\DeclareMathSymbol{I}{\mathalpha}{operators}{`I}
\DeclareMathSymbol{J}{\mathalpha}{operators}{`J}
\DeclareMathSymbol{K}{\mathalpha}{operators}{`K}
\DeclareMathSymbol{L}{\mathalpha}{operators}{`L}
\DeclareMathSymbol{M}{\mathalpha}{operators}{`M}
\DeclareMathSymbol{N}{\mathalpha}{operators}{`N}
\DeclareMathSymbol{O}{\mathalpha}{operators}{`O}
\DeclareMathSymbol{P}{\mathalpha}{operators}{`P}
\DeclareMathSymbol{Q}{\mathalpha}{operators}{`Q}
\DeclareMathSymbol{R}{\mathalpha}{operators}{`R}
\DeclareMathSymbol{S}{\mathalpha}{operators}{`S}
\DeclareMathSymbol{T}{\mathalpha}{operators}{`T}
\DeclareMathSymbol{U}{\mathalpha}{operators}{`U}
\DeclareMathSymbol{V}{\mathalpha}{operators}{`V}
\DeclareMathSymbol{W}{\mathalpha}{operators}{`W}
\DeclareMathSymbol{X}{\mathalpha}{operators}{`X}
\DeclareMathSymbol{Y}{\mathalpha}{operators}{`Y}
\DeclareMathSymbol{Z}{\mathalpha}{operators}{`Z}

% Raccourci displaystyle + hack :-)
\newcommand{\dps}{\displaystyle}%
\newcommand{\dsp}{\displaystyle}%
\newcommand{\disp}{\displaystyle}%
\everymath{\displaystyle}%

% Mots usuels en mode math
\newcommand{\mtext}[1]{\quad\text{#1}\quad}%
\newcommand{\et}{\mtext{et}}%
\newcommand{\ou}{\mtext{ou}}%
\newcommand{\si}{\mtext{si}}%

% Flèches
\newcommand{\tv}{\shortrightarrow}% tend vers
\renewcommand{\to}{\shortrightarrow}% tend vers
\newcommand{\suit}{\hookrightarrow}% X suit la loi...
\newcommand{\dans}{\longrightarrow}% f:\R\dans\R
\newcommand{\donne}{\longmapsto}% f:x\donne 2x+3
\newcommand{\ppv}{\leftarrow}% flèche <-- d'affectation "prend pour valeur"
\newcommand{\ech}{\leftrightarrow}% double flèche <--> : échange/swap

% Vecteurs
% \vv{AB} en utilisant l'extension \usepackage[e]{esvect}

% Norme et valeur absolue
\newcommand{\abs}[1]{\left\lvert#1\right\rvert}%
\newcommand{\norme}[1]{\left\lVert#1\right\rVert}%

% Complexes
\renewcommand{\Re}{\operatorname{Re}}
\renewcommand{\Im}{\operatorname{Im}}
\renewcommand{\bar}{\overline}

% Matrices
\newcommand{\trans}[1]{{\vphantom{#1}}^{\mathit{t}}\!{#1}}%

% Coefficient binomial
\newcommand{\cb}[2]{\binom{#2}{#1}}%

% Matrice augmentée
\makeatletter
\renewcommand*\env@matrix[1][*\c@MaxMatrixCols c]{%
  \hskip -\arraycolsep
  \let\@ifnextchar\new@ifnextchar
  \array{#1}}
\makeatother

% Parallèles et perpendiculaires
\newcommand{\para}{\sslash}%
% perp pour perpendiculaire

% Intervalles
\newcommand{\intervalle}[4]{\mathchoice%
{\left#1#2\mathclose{}\mathpunct{},#3\right#4}% mode \displaystyle
{\mathopen{#1}#2\mathclose{}\mathpunct{},#3\mathclose{#4}}% mode \textstyle
{\mathopen{#1}#2\mathclose{}\mathpunct{},#3\mathclose{#4}}% mode \scriptstyle
{\mathopen{#1}#2\mathclose{}\mathpunct{},#3\mathclose{#4}}% mode \scriptscriptstyle
}%

\newcommand{\intff}[2]{\intervalle{[}{#1}{#2}{]}}%
\newcommand{\intof}[2]{\intervalle{]}{#1}{#2}{]}}%
\newcommand{\intfo}[2]{\intervalle{[}{#1}{#2}{[}}%
\newcommand{\intoo}[2]{\intervalle{]}{#1}{#2}{[}}%

% Ancienne configuration
%\newcommand{\intervalle}[4]{\mathopen{#1}#2\mathclose{}\mathpunct{},#3\mathclose{#4}}%
%\newcommand{\intoo}[2]{\ensuremath{\,\left]  #1 \,, #2  \right[\, }}%
%\newcommand{\intof}[2]{\ensuremath{\,\left]  #1 \,, #2  \right]\, }}%
%\newcommand{\intfo}[2]{\ensuremath{\,\left[  #1 \,, #2  \right[\, }}%
%\newcommand{\intff}[2]{\ensuremath{\,\left[  #1 \,, #2  \right]\, }}%

% Intervalles entiers
\newcommand{\intn}[2]{\intervalle{\llbracket}{#1}{#2}{\rrbracket}}%

% Ensembles et Probabilités
\newcommand{\vide}{\varnothing}% ensemble vide
\newcommand{\union}{\cup}%
\newcommand{\inter}{\cap}%
\newcommand{\Union}{\bigcup}%
\newcommand{\Inter}{\bigcap}%
\newcommand{\compl}{\complement}% complémentaire
\newcommand{\inclus}{\subseteq}% inclus : je n'aime pas \subset je préfère \subseteq...
\newcommand{\inclusstrict}{\subsetneq}% inclus au sens strict ...
\newcommand{\contient}{\supseteq}% contient
\newcommand{\contientstrict}{\supsetneq}% contient au sens strict ...
\newcommand{\prive}{\setminus}% privé de ...
\renewcommand{\P}{\mathrm{P}} % probabilité
\newcommand{\V}{\mathrm{V}} % variance
\newcommand{\E}{\mathrm{E}} % espérance

% ensemble des ... tels que ...
\newcommand{\enstq}[2]{\left\{#1\,\;\middle|\;\,#2\right\}}%

% Pointillés anti-diagonale
\newcommand{\adots}{\mathinner{\mkern2mu\raise 1pt\hbox{.}\mkern3mu\raise 4pt\hbox{.}\mkern1mu\raise 7pt\hbox{.}}}%

% Partie entière
\newcommand{\ent}[1]{\left\lfloor#1\right\rfloor}% partie entière (première notation)
\newcommand{\Ent}[1]{\textrm{Ent}\mathopen{}\left(#1\right)}% partie entière (deuxième notation)

% Angle
\renewcommand{\angle}{\widehat}%

% Limites
\newcommand{\iy}{\infty}% 
\newcommand{\ii}{\infty}%
\newcommand{\zp}{0^{+}}%
\newcommand{\zm}{0^{-}}%

% Encadrement d'une formule
%\begin{empheq}[box=\fbox]{equation*}
% ...    
%\end{empheq}

% Couleurs
% http://www.latextemplates.com/svgnames-colors
\definecolor{bleu1}{HTML}{000080}%
\definecolor{grispale}{RGB}{245 245 245}%
\definecolor{bistre}{rgb}{.75 .50 .30}%
\definecolor{grisclair}{gray}{0.8}%
\definecolor{bleuclair}{rgb}{0.7, 0.7, 1.0}%
\definecolor{rosepale}{rgb}{1.0, 0.7, 1.0}%

% Fluos !
\newcommand{\fluo}[1]{\sethlcolor{rosepale}\hl{#1}}%

% Lettres calligraphiées
\newcommand{\Cf}{\mathscr{C}_f}%
\newcommand{\Df}{\mathscr{D}_f}%
\newcommand{\Cg}{\mathscr{C}_g}%
\newcommand{\Dg}{\mathscr{D}_g}%
\newcommand{\Ch}{\mathscr{C}_h}%
\newcommand{\Dh}{\mathscr{D}_h}%

% Degré
\newcommand{\Degre}{\ensuremath{^\circ}}

% Lettres grecques
\renewcommand{\epsilon}{\varepsilon}%
\renewcommand{\phi}{\varphi}%

% Mots usuels
\newcommand{\ie}{\;\textit{i.e.}\;\xspace}
\newcommand{\cad}{c'est--à--dire\xspace}%
\newcommand{\pourcent}{\unskip~\%\xspace}%
\newcommand{\ssi}{si et seulement si\xspace}%
\newcommand{\eve}{événement\xspace}%
\newcommand{\eves}{événements\xspace}%
\newcommand{\sev}{sous-espace vectoriel\xspace}%
\newcommand{\ipp}{intégration par parties\xspace}%
\newcommand{\iaf}{inégalité des accroissements finis\xspace}%
\newcommand{\tvi}{théorème des valeurs intermédiaires\xspace}%
\newcommand{\fpt}{formule des probabilités totales\xspace}%
\newcommand{\fpc}{formule des probabilités composées\xspace}%
\newcommand{\sce}{système complet d'événements\xspace}%
\newcommand{\srld}{suite récurrence linéaire d'ordre $2$\xspace}%
\newcommand{\sag}{suite arithmético-géométrique\xspace}%

% Guillements français
\newcommand{\guill}[1]{%
\og{}#1\fg{}}%


% Inégalités
\renewcommand{\leq}{\leqslant}%
\renewcommand{\geq}{\geqslant}%
\renewcommand{\le}{\leqslant}%
\renewcommand{\ge}{\geqslant}%
\newcommand{\pg}{\geqslant}%
\newcommand{\pp}{\leqslant}%

% Environ
\newcommand{\environ}{\simeq}%
\renewcommand{\approx}{\simeq}%


% Tableau de variations en TikZ
\usepackage{tikz,tkz-tab}%
\definecolor{fondpaille}{rgb}{1,1,1}%

% Aire d'une figure géométrique
\newcommand{\aire}{\text{aire}}%

% Paramétrage de quelques variables
\setlength{\columnsep}{1cm}%
\setlength{\columnseprule}{0.4pt}%
\setlength{\parindent}{0pt}%

% Mathématiciens
\newcommand{\GJ}{Gauss\,--\,Jordan\xspace}%
\newcommand{\KH}{König\,--\,Huygens\xspace}%

% Siècle en lettres romains
\newcommand{\siecle}[1]{\textsc{\romannumeral #1}\textsuperscript{e}~si\`ecle}%

% Couleurs jeu de carte
\newcommand{\pique}{\ding{171}}%
\newcommand{\coeur}{\ding{170}}%
\newcommand{\carreau}{\ding{169}}%
\newcommand{\trefle}{\ding{168}}%

% Exercices (fiche)
\renewcommand*{\hrulefill}[2][0pt]{\leavevmode \leaders \hbox to 1pt{\rule[#1]{1pt}{#2}} \hfill \kern 0pt}%

\newcounter{numexercice}%

\newenvironment{exercice}{\stepcounter{numexercice}\ovalbox{\textbf{\thenumexercice}}\hrulefill[3pt]{0.5pt}\par\medskip\nopagebreak[4]}{\medskip}

% Trait de la largeur de la feuille
\newcommand{\trait}{\hbox{\raisebox{0.4em}{\vrule depth 0pt height 0.4pt width \textwidth}\linebreak}}%

\newcommand{\demitrait}{\hbox{\raisebox{0.4em}{\vrule depth 0pt height 0.4pt width 0.48\textwidth}\linebreak}}%

\newcommand{\LV}{Lycée Le Verrier, Saint\,--\,Lô}%
%\newcommand{\itb}{\item[\textbullet]}%
\newcommand{\itb}{\item}%
%\newcommand{\Gaffe}{\ding{54}\ding{54}\ding{54}\quad}%
\newcommand{\gaffe}{\ding{56}\ding{56}\ding{56}\quad}%


% Fichier de style stage2.sty [UTF8]
% Copyleft Laurent Bretonnière, laurent.bretonniere@gmail.com
% Version du 16/03/2015

\usepackage{mathtools}%	
\usepackage{fancybox}%
\usepackage{lastpage}%

\usepackage{fancyhdr}%
\renewcommand{\headrulewidth}{0.8pt}%
\renewcommand{\footrulewidth}{0.8pt}%

\usepackage[tikz]{bclogo}%
\renewcommand\bcStyleTitre[1]{\normalsize\textbf{#1}\smallskip}%
\renewcommand\logowidth{0pt}%

\newcommand{\fin}{\begin{center}%
$\clubsuit\clubsuit\clubsuit$%
\end{center}}%

\newcommand{\un}{\ding{192}\xspace}%
\newcommand{\deux}{\ding{193}\xspace}%
\newcommand{\trois}{\ding{194}\xspace}%
\newcommand{\quatre}{\ding{195}\xspace}%
\newcommand{\cinq}{\ding{196}\xspace}%
\newcommand{\six}{\ding{197}\xspace}%
\newcommand{\sept}{\ding{198}\xspace}%
\newcommand{\huit}{\ding{199}\xspace}%
\newcommand{\neuf}{\ding{200}\xspace}%

\setlength{\headheight}{15pt}%

%*********************************************************************************************
% Cours
%*********************************************************************************************

\usepackage[Lenny]{fncychap}%
\ChNumVar{\fontsize{76}{80}\usefont{OT1}{pzc}{m}{n}\selectfont}%
\ChTitleVar{\raggedleft\Huge\sffamily\bfseries}%

\renewcommand{\thesection}{\Roman{section})}%
\renewcommand{\thesubsection}{\arabic{subsection})}%
\renewcommand{\thesubsubsection}{\alph{subsubsection})}%

%*********************************************************************************************
% Environnements prédéfinis BCLOGO
%*********************************************************************************************

%% Lemme
\newenvironment{lem}{\begin{bclogo}[couleurBord=black!50,arrondi=0.1,logo=\hspace{17pt},barre=none]{Lemme :}}{\end{bclogo}\medskip}%

%% Proposition
\newenvironment{prop}[1][]{\begin{bclogo}[couleurBord=black!50,arrondi=0.1,logo=\hspace{17pt},barre=none]{Proposition :~#1}}{\end{bclogo}\medskip}%

%% Théorème
\newlength{\textlarg}
\settowidth{\textlarg}{~}
\newenvironment{theo}[1][\hspace{-\textlarg} :]{\begin{bclogo}[couleur=black!5,couleurBord=black!50,arrondi=0.1,logo=\hspace{17pt}, barre=none]{Théorème~#1}}{\end{bclogo}\medskip}%

\newenvironment{theon}[1][]{\begin{bclogo}[couleur=black!5,couleurBord=black!50,arrondi=0.1,logo=\hspace{17pt}, barre=none]{Théorème :~#1}}{\end{bclogo}\medskip}%

%% Corollaire
\newenvironment{coro}[1][]{\begin{bclogo}[couleurBord=black!50,arrondi=0.1,logo=\hspace{17pt},barre=none]{Corollaire :~#1}}{\end{bclogo}\medskip}%

%% Définition(s)

\newenvironment{defi}{\begin{bclogo}[couleurBord=black!50,arrondi=0.1,logo=\hspace{17pt}, barre=none]{Définition :}}{\end{bclogo}\medskip}%

\newenvironment{defis}{\begin{bclogo}[couleurBord=black!50,arrondi=0.1,logo=\hspace{17pt}, barre=none]{Définitions :}}{\end{bclogo}\medskip}%

%% Preuve
\newenvironment{pf}{\renewcommand\logowidth{17pt}\begin{bclogo}[noborder=true,logo=\hspace{17pt},couleurBarre=black!25,epBarre=3.5]{Preuve :}}{\hspace*{\fill}$\Box$\end{bclogo}\smallskip\renewcommand\logowidth{0pt}}%

%\blacksquare

%% Notation
\newenvironment{nota}{\begin{bclogo}[couleurBord=black!50,arrondi=0.1,logo=\hspace{17pt},barre=none]{Notation :}}{\medskip}%

%% Exercice et Exercice-type
\newenvironment{exo}{$\circledast$ \quad\textsc{\underline{exercice} :}~}{\hspace*{\fill}$\circledast$\vskip 8pt}
\newenvironment{type}{$\blacktriangleright$ \quad\textsc{exercice-type :}~}{\hspace*{\fill}$\blacktriangleleft$\vskip 8pt}

%% Exemple(s)
\newenvironment{exem}{\textbf{Exemple :}~}{\medskip}
\newenvironment{exems}{\textbf{Exemples :}~}{\medskip}

%% Remarque(s)
\newenvironment{rem}{\textbf{Remarque :}~}{\medskip}
\newenvironment{rems}{\textbf{Remarques :}~}{\medskip}

%% Rappel(s)
\newenvironment{rap}{\textbf{Rappel :}~}{\medskip}
\newenvironment{raps}{\textbf{Rappels :}~}{\medskip}

%% Cas particulier(s)
\newenvironment{cp}{\textbf{Cas particulier :}~}{\medskip}
\newenvironment{cps}{\textbf{Cas particuliers :}~}{\medskip}

%% Application
\newenvironment{appli}{\textbf{Application :}~}{\medskip}%{\medskip} 

%******************************************

%Permet le code python sur lateX
\usepackage{minted}
\usemintedstyle{lovelace}

%box exercice
\usepackage{tcolorbox}
\newtcolorbox{mybox}[1]{colback=yellow!5!,colframe=yellow!50!black,colbacktitle=yellow!75!black,fonttitle=\bfseries,
title=#1}

%%Propriété
\newenvironment{pro}[1][]{\begin{bclogo}[couleurBord=black!50,arrondi=0.1,logo=\hspace{17pt},barre=none]{Propriété :~#1}}{\end{bclogo}\medskip}%




\usepackage[left=2cm,right=2cm,top=2cm,bottom=2cm]{geometry}
\def\Oij{$\left(\text{O},~\vec{i},~\vec{j}\right)$}
\usepackage{fancyhdr}
\usepackage{MnSymbol,wasysym}

%Permet le code python sur lateX
\usepackage{minted}
\usemintedstyle{lovelace}

%Permet de mettre les coordonnées d'un vecteur
\newcommand*{\Coord}[3]{% 
  \ensuremath{\overrightarrow{#1}\, 
    \begin{pmatrix} 
      #2\\ 
      #3 
    \end{pmatrix}}}

\begin{document}
\textbf{1SPEmaths} \hfill \textbf{Produit scalaire} \hfill Lycée Jean Rostand\\
\trait 

\subsection*{Exercice n°1}

On considère le carré ABCD de centre O et de côté 8.

\begin{center}
  
\begin{tikzpicture}[scale=0.75]
\draw  (0,0)-- (0,4);
\draw  (0,4)-- (4,4);
\draw  (4,4)-- (4,0);
\draw  (0,0)-- (4,4);
\draw  (0,4) -- (4,0);
\draw  (0,0) -- (4,0);

\draw  (-0.3,0) node {$A$};
\draw  (-0.3,4) node {$D$};
\draw  (4.3,4) node {$C$};
\draw (4.3,0) node {$B$};
\draw (2,1.5) node {$O$};
\end{tikzpicture}
\end{center}
Calculer les produits scalaire suivants:
\begin{enumerate}
\begin{minipage}[t]{0.4\linewidth}
\item $\vv{AB}.\vv{AO}$
\item $\vv{AB}.\vv{AD}$
\end{minipage}
\begin{minipage}[t]{0.4\linewidth}
\item $\vv{AB}.\vv{CD}$
\item $\vv{BO}.\vv{OD}$
\end{minipage}
\begin{minipage}[t]{0.4\linewidth}
\item $\vv{OB}.\vv{DO}$
\end{minipage}

\end{enumerate}

\subsection*{Exercice n°A}


On considère le carré ABCD de centre O et de côté 6.

\begin{center}
  
\begin{tikzpicture}[scale=0.75]
\draw  (0,0)-- (0,4);
\draw  (0,4)-- (4,4);
\draw  (4,4)-- (4,0);
\draw  (0,0)-- (4,4);
\draw  (0,4) -- (4,0);
\draw  (0,0) -- (4,0);

\draw  (-0.3,0) node {$A$};
\draw  (-0.3,4) node {$D$};
\draw  (4.3,4) node {$C$};
\draw (4.3,0) node {$B$};
\draw (2,1.5) node {$O$};
\end{tikzpicture}
\end{center}
Calculer les produits scalaire suivants:
\begin{enumerate}
\begin{minipage}[t]{0.4\linewidth}
\item $\vv{AD}.\vv{AO}$
\item $\vv{OB}.\vv{OD}$
\end{minipage}
\begin{minipage}[t]{0.4\linewidth}
\item $\vv{BO}.\vv{BC}$
\item $\vv{AO}.\vv{OC}$
\end{minipage}
\begin{minipage}[t]{0.4\linewidth}
\item $\vv{AB}.\vv{OD}$
\end{minipage}

\end{enumerate}


\subsection*{Exercice n°2}

ABCD est un rectangle (avec $AB=4$ et $AD=3$ ) de centre $F$ et $E$ est le symétrique de $F$ par rapport à (BC) .

\begin{center}
    
\begin{tikzpicture}[scale=0.75]
\draw  (0,0)-- (0,3);
\draw  (0,3)-- (4,3);
\draw  (4,3)-- (4,0);
\draw  (0,0)-- (4,3);
\draw (0,3)--(4,0);
\draw (0,0) -- (4,0);
\draw  (4,3) -- (6,1.5);
\draw  (4,0) -- (6,1.5);

\draw  (-0.3,0) node {$A$};
\draw  (-0.3,3) node {$D$};
\draw  (4.5,3) node {$C$};
\draw (4.5,0) node {$B$};
\draw (2,2) node {$O$};
\draw (6.3,1.8) node {$E$};
\end{tikzpicture}
\end{center}



Calculer les produits scalaires suivants:


\begin{enumerate}
\begin{minipage}[t]{0.4\linewidth}
\item $\vv{CF}.\vv{CD}$
\item $\vv{BA}.\vv{BE}$
\end{minipage}
\begin{minipage}[t]{0.4\linewidth}
\item $\vv{AF}.\vv{AB}$
\item $\vv{AB}.\vv{BE}$
\end{minipage}
\begin{minipage}[t]{0.4\linewidth}
\item $\vv{BF}.\vv{DC}$
\item $\vv{AF}.\vv{BE}$
\end{minipage}

\end{enumerate}

\subsection*{Exercice n°3}

Dans chacun des cas suivants, calculer le produit scalaire de $\vv{u}$ par $\vv{v}$:

\begin{enumerate}
\begin{minipage}[t]{0.5\linewidth}
\item $\norme{\vv{u}}=2$, $\norme{\vv{v}}=3$ et $(\vv{u};\vv{v})=60^\circ $
\item $\norme{\vv{u}}=1$, $\norme{\vv{v}}=4$ et $(\vv{u};\vv{v})=\dfrac{\pi}{3} $
\end{minipage}
\begin{minipage}[t]{0.5\linewidth}
\item $\norme{\vv{u}}=8$, $\norme{\vv{v}}=\sqrt{2}$ et $(\vv{u};\vv{v})=\dfrac{3\pi}{4} $
\item $\norme{\vv{u}}=5$, $\norme{\vv{v}}=\sqrt{3}$ et $(\vv{u};\vv{v})=135^\circ $
\end{minipage}
\end{enumerate}


\subsection*{Exercice n°4}

Déterminer une valeur en radian de l'angle de vecteur $(\vv{u};\vv{v})$ dans chacun des cas suivants:

\begin{enumerate}
    \item $\norme{\vv{u}}=6$, $\norme{\vv{v}}=2$ et $(\vv{u}.\vv{v})=-6 $
    \item $\norme{\vv{u}}=2$, $\norme{\vv{v}}=\sqrt{3}$ et $(\vv{u}.\vv{v})=\sqrt{6} $
\end{enumerate}

\subsection*{Exercice n°5}

On considère le carré ABCD de côté $5$.\\ Calculer le produit scalaire $\vv{AB}.\vv{AC}$\\
On passera par la définition avec le cosinus et on pourra réaliser un dessin à main levée pour visualiser la situation.

\subsection*{Exercice n°6}

Dans un repère \Oij{}:

\begin{enumerate}
    \item On pose $\Coord{u}{2}{-3}$ et $\Coord{v}{4}{\frac{5}{3}}$. Calculer $\vv{u}.\vv{v}$ 
    \item On pose $\Coord{w}{5}{-3}$ et $\Coord{t}{2}{y}$. Déterminer $y$ sachant que $\vv{w}.\vv{t}=1$ 
\end{enumerate}

\subsection*{Exercice n°7}

Soient les vecteurs $\Coord{u}{-2}{3}$ et 
$\Coord{v}{-1}{-5}$. Calculer:

\begin{enumerate}
\begin{minipage}[t]{0.4\linewidth}
\item $\vv{u}.\vv{v}$

\end{minipage}
\begin{minipage}[t]{0.4\linewidth}
\item $(4\vv{u}).\vv{v}$

\end{minipage}
\begin{minipage}[t]{0.4\linewidth}
\item $(\vv{u}-\vv{v})(\vv{u}+\vv{v})$
\end{minipage}
\end{enumerate}

\subsection*{Exercice n°8}

Soient les vecteurs $\Coord{u}{2}{1}$ et 
$\Coord{v}{-3}{-1}$ et $\Coord{w}{1}{4}$. Calculer:

\begin{enumerate}
\begin{minipage}[t]{0.3\linewidth}
\item $\vv{u}.\vv{v}$

\end{minipage}
\begin{minipage}[t]{0.3\linewidth}
\item $\vv{w}.\vv{v}$

\end{minipage}
\begin{minipage}[t]{0.3\linewidth}
\item $\vv{u}.(\vv{v}+\vv{w})$
\end{minipage}

\begin{minipage}[t]{0.3\linewidth}
\item $(-2\vv{u}).\vv{v}+3(\vv{v}.\vv{w})$
\end{minipage}

\end{enumerate}


\subsection*{Exercice n°9}

On considère les points $A(-2;3)$, $B(-1;-2)$ , $C(0,4)$ et $D(2;5)$. Calculer:

\begin{enumerate}
\begin{minipage}[t]{0.2\linewidth}
\item $\vv{AB}.\vv{BC}$

\end{minipage}
\begin{minipage}[t]{0.2\linewidth}
\item $\vv{CB}.\vv{BD}$

\end{minipage}
\begin{minipage}[t]{0.2\linewidth}
\item $\vv{AB}.\vv{CD}$
\end{minipage}

\begin{minipage}[t]{0.2\linewidth}
\item $\vv{BA}.\vv{AD}$
\end{minipage}

\end{enumerate}

\subsection*{Exercice n°B}

Dans un repère orthonormé du plan, on considère les point A,B et C dont les coordonnées respectives sont $(-2;-2)$, $(3;1)$ et $(-1;2)$.
\begin{enumerate}
    \item 
    \begin{enumerate}
        \item Calculer les coordonnées $\vv{AB}$ et $\vv{AC}$
        \item En déduire la valeur du produit scalaire $\vv{AB}.\vv{AC}$
    \end{enumerate}
    \item
    \begin{enumerate}
        \item Calculer les distances $AB$ et $AC$
        \item En déduire une valeur de l'angle $\widehat{BAC}$ en radian
    \end{enumerate}

\end{enumerate}

\subsection*{Exercice n°10} 

On se situe dans un repère orthonormé du plan

\begin{enumerate}
    \item Montrer que les vecteurs $\Coord{u}{-3}{4}$ et $\Coord{v}{-8}{-6}$ sont orthogonaux.
    \item On donne les points $A(-3;-2)$ et $B(1;3)$ et le vecteur $\Coord{u}{-5}{4}$. \\
    Montrer que $\vv{AB}$ et $\vv{u}$ sont orthogonaux.
\end{enumerate}

\subsection*{Exercice n°C} 

Dans un repère orthonormé du plan, on donne $A(5;3)$, $B(8;-5)$, $C(-1;0)$ et $D(3;1,5)$.\\
Montrer que $(AB)$ et $(CD)$ sont perpendiculaires.


\subsection*{Exercice n°11} 
Dans les cas suivants:

\begin{enumerate}
    \item Dire si les vecteurs $\vv{u}$ et $\vv{v}$ sont orthogonaux:
    \begin{enumerate}
        \item $\Coord{u}{-1}{3}$ et $\Coord{v}{3}{-1}$
        \item $\Coord{u}{2}{4}$ et $\Coord{v}{-6}{3}$
    \end{enumerate}
    \item Dire si les droites $(AB)$ et $(CD)$ sont perpendiculaires:
    \begin{enumerate}
        \item $A(2;-3)$ , $B(-1;-1)$, $C(5;-3)$ et $D(2;1)$
        \item $A(-1;-2)$ , $B(-2;-4)$, $C(7;-1)$ et $D(3;1)$
    \end{enumerate}
    \item Déterminer la ou les valeurs de $a$ pour que $\vv{u}$ et $\vv{v}$ soient orthogonaux:
    \begin{enumerate}
        \item $\Coord{u}{-5}{4}$ et $\Coord{v}{1}{a}$
        \item $\Coord{u}{2}{a+1}$ et $\Coord{v}{a+5}{3}$
        \item $\Coord{u}{a}{-3+a}$ et $\Coord{v}{2}{a}$
    \end{enumerate}
\end{enumerate}

\newpage
\subsection*{Exercice n°12} 

Donner un vecteur directeur pour chacune des droites suivantes et en déduire qu'elles sont perpendiculaires.

\begin{enumerate}
    \item $\mathscr{D}_{1}:2x-3y+4=0$ et $\mathscr{D}_2:3x+2y-1=0$
    \item $\mathscr{D}_{1}:x-y+3=0$ et $\mathscr{D}_2:2x+2y-1=0$
    \item $\mathscr{D}_{1}:y=3x+1$ et $\mathscr{D}_2:-x+3y-1=0$
\end{enumerate}

\begin{raps}

Une droite $\mathscr{D}$ d'équation cartésienne $ax+by+c=0$ a pour vecteur directeur $\Coord{u}{-b}{a}$.\\

Une droite d'équation réduite $y=ax+b$ a pour vecteur directeur $\Coord{v}{1}{a}$.
\end{raps}


\subsection*{Exercice n°13} 

Déterminer une équation de la médiatrice du segment $[AB]$ avec $A(3,-5)$ et $B(1;4)$ dans un repère orthonormé \Oij{}
\textit{On pourra faire un dessin à main levée pour visualiser la situation donnée}

\subsection*{Exercice n°14} 

On considère deux carrés $ABCD$ et $BEFG$ disposés comme sur la figure ci-dessous tel que $AB=1$ et $BE=a$


\begin{center}
  
\begin{tikzpicture}[scale=1]
\draw  (0,0)-- (0,1);
\draw  (0,1)-- (1,1);
\draw  (1,1)-- (1,0);
\draw  (0,0) -- (1,0);
\draw  (1,0)-- (1,2);
\draw  (1,2)--(3,2);
\draw (3,2)-- (3,0);
\draw (3,0)--(1,0);
\draw[dash pattern=on 1mm off 1mm] (-1,-2)--(2,4);
\draw[dash pattern=on 1mm off 1mm] (5,-1)--(-1,2);

\draw  (0,0) node[left] {$A$};
\draw  (0,1) node [left]{$D$};
\draw  (1,1) node[above,right] {$C$};
\draw  (1,0) node[below] {$B$};
\draw  (3,0) node[right] {$E$};
\draw  (1,2) node[left] {$G$};
\draw  (3,2) node[right] {$F$};
\end{tikzpicture}
\end{center}

\textbf{Partie A: avec coordonnées}
\begin{enumerate}
    \item Dans le repère $(A;B;D)$ , donner les coordonnées de tous les points de la figure.
    \item  Démontrer que les droites $(AG)$ et $(CE)$ sont perpendiculaires.
\end{enumerate}

\textbf{Partie B: sans coordonnées}

\begin{enumerate}
    \item Développer le produit scalaire $(\vv{AB}+\vv{BG}).(\vv{CB}+\vv{BE})$
    \item En déduire que $\vv{AG}.\vv{CE}=0$ et conclure.
\end{enumerate}

\newpage
\subsection*{Exercice n°D} 

Soit $a$ un nombre réel positif. On considère le rectangle $ABCD$ tel que $AB=a$ et $AD=\dfrac{\sqrt{2}}{2}a$.
On note $I$ milieu de $[CD]$.


\begin{center}
  
\begin{tikzpicture}[scale=1]
\draw  (0,3)-- (4,3);
\draw  (4,3)-- (4,0);
\draw  (4,0)-- (0,0);
\draw  (0,0) -- (0,3);
\draw  (0,3)-- (4,0);
\draw  (2,0)--(4,3);



\draw  (0,0) node[left] {$D$};
\draw  (0,3) node [left]{$A$};
\draw  (4,3) node[right] {$B$};
\draw  (4,0) node[right] {$C$};
\draw  (2,0) node[below] {$I$};
\draw (1,0) node {$//$};
\draw (3,0) node {$//$};
\end{tikzpicture}
\end{center}

En se servant uniquement des propriétés algébriques, démontrer que les droites $(AC)$ et $(BI)$ sont perpendiculaires.

\subsection*{Exercice n°E} 

On considère la figure ci-dessous.

\begin{enumerate}
    \item Calculer la longueur $BC$
    \item Calculer la longueur $CD$
    \item En déduire la mesure de l'angle $\widehat{BDC}$ en degrés arrondie au dixième 
\end{enumerate}


\begin{center}
  
\begin{tikzpicture}[scale=0.75]
\draw  (0,0)-- (4.48,4.32);
\draw  (4.48,4.32)-- (11,0);
\draw  (4.48,4.32)-- (7,0);
\draw  (0,0) -- (7,0);
\draw  (7,0)-- (11,0);




\draw  (0,0) node[left] {$D$};
\draw  (7,0) node [below]{$A$};
\draw  (11,0) node[right] {$B$};
\draw  (4.48,4.32) node[above] {$C$};
\draw  (7,0) ++(0:1) arc(0:120:1);
\draw (7.4,0.35) node {$120^\circ$};
\draw (3.5,0) node[below]{$7$};
\draw (6,2) node[right]{$5$};
\draw (9,0) node[below]{$4$};
\end{tikzpicture}
\end{center}


\subsection*{Exercice n°15} 

On considère les points A, B et C tels que $AB=3$, $AC=4$ et $\widehat{BAC}=120^\circ$.\\
Déterminer la longueur $BC$.\\
On pourra réaliser une figure à main levée pour visualiser la situation.


\subsection*{Exercice n°16} 

On considère les points M,N et P tels que $MN=5$, $NP=7$ et $\widehat{MNP}=61^\circ$.\\
Déterminer la longueur MP.

\subsection*{Exercice n°17} 

Soit un triangle $EFG$ tel que $EF=7$, $FG=6$ et $EG=11$.\\
Déterminer la valeur en degrés et arrondie au dixième de l'angle $\widehat{EFG}$

\subsection*{Exercice n°18} 

Soit un triangle $EFG$ tel que $EF=5$, $FG=8$ et $\widehat{EFG}=60^\circ$.\\

\begin{enumerate}
    \item Déterminer la valeur exaxte de $EG$, puis une valeur approchée  arrondie au dixième.
    \item Déterminer une valeur approchée de $\widehat{FGE}$ arrondie à l'unité.
\end{enumerate}

\subsection*{Exercice n°19} 

On donne les points A et B tels que $AB=12$ et $I$ le milieu du segment $[AB]$.\\
Déterminer l'ensemble des points $M$ du plan vérifiant $\vv{MA}.\vv{MB}=4$

\subsection*{Exercice n°20} 

On donne les points C et D tels que $CD=10$ et $H$ le milieu du segment $[CD]$.\\
Déterminer l'ensemble des points $M$ du plan vérifiant $\vv{MC}.\vv{MD}=-9$

\subsection*{Exercice n°21} 

On considère les points $A(-2;-3)$ et $B(-1;4)$.\\

\begin{enumerate}
    \item Calculer la longueur $AB$
    \item Déterminer les coordonnées du milieu du segment $[AB]$
    \item Déterminer l'ensemble des points $M$ du plan vérifiant $\vv{MA}.\vv{MB}=0$
\end{enumerate}

\subsection*{Exercice n°22} 

On considère un triangle $ABC$ et $A'$ est le milieu du segment $[BC]$.\\
Déterminer l'ensemble des points $M$ du plan vérifiant $\vv{MA}.(\vv{MB}+\vv{MC})=0$

\subsection*{Exercice n°23} 

On considère deux points $C$ et $V$ tels que $CV=8$.\\
Déterminer l'ensemble des points $M$ tels que $\vv{CM}.\vv{VM}=-4$




\end{document}

